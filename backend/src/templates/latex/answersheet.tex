\documentclass[UTF8,unicode]{ctexrep}

% 定义长度 
\def\zbj{1.5cm}%定义左边距
\def\ybj{1cm}%定义右边距
\def\sbj{1cm}%定义上边距
\def\xbj{1cm}%定义下边距
\def\ljjy{0.45cm}%定义栏间距1
\def\ljje{0.6cm}%定义栏间距2
\def\khws{6}%定义考号位数
\def\lk{(\textwidth-\ljjy-\ljje)/3}%定义栏宽
\def\gzk{(\lk/2-0.4cm)/\khws}%定义格子款
\def\ttkc{0.36}%定义填涂框长
\def\ttkg{0.192}%定义填涂框高

%定义选择题填涂格
\newcommand{\dxtt}[1]{\makebox[1.4em]{\raisebox{-2.5pt}{
	\begin{tikzpicture}[BrickRed]
		\draw (0,0)rectangle(\ttkc,\ttkg);
		\draw(\ttkc/2,\ttkg/2-0.01)node{\zihao{-6}\textsf{#1}};
	\end{tikzpicture}}}}
\newcommand{\sxtt}{\raisebox{0pt}{\dxtt{A} \dxtt{B} \dxtt{C} \dxtt{D}}}

%页面设置
\usepackage{geometry}
\geometry{b4paper,left=\zbj,right=\ybj,top=\sbj,bottom=\xbj,headsep=0.5cm,footskip=0.5cm,landscape}
\usepackage{fancyhdr}
\pagestyle{fancy}
\fancyhf{}
\renewcommand\headrulewidth{0pt}

%加载宏包
\usepackage{amsmath,amssymb}
\usepackage[dvipsnames,table]{xcolor}
\usepackage{pgf,tikz,multicol,calc}
\usepackage{varwidth}

%定义填空题横线
\newcommand{\tk}[1][2.5]{\,\underline{\mbox{\hspace{#1 cm}}}\,}

\begin{document}

\begin{tikzpicture}[remember picture,overlay]
	\coordinate (A1) at ([shift={(\zbj,-\sbj)}]current page.north west);
	\coordinate (B1) at ([shift={({\zbj+\lk},-\sbj)}]current page.north west);
	\coordinate (C1) at ([shift={({\zbj+\lk},\xbj)}]current page.south west);
	\coordinate (D1) at ([shift={(\zbj,\xbj)}]current page.south west);
	\coordinate (A2) at ([shift={(\ljjy,0)}]B1);
	\coordinate (B2) at ([shift={({\lk},0)}]A2);
	\coordinate (C2) at ([shift={(0,-\textheight)}]B2);
	\coordinate (D2) at ([shift={(\ljjy,0)}]C1);
	\coordinate (A3) at ([shift={(\ljje,0)}]B2);
	\path([shift={(-\ybj,-\sbj)}]current page.north east) coordinate (B3);
	\path([shift={(-\ybj,\xbj)}]current page.south east) coordinate (C3);
	\coordinate (D3) at ([shift={(\ljje,0)}]C2);
	\path([shift={(0.2,-2.1)}]A1) coordinate (A0);
	\path([shift={({\lk/2-0.4cm},-0.7)}]A0) coordinate (C0);
	\coordinate (khT1) at ([shift={({0.5*\gzk-0.18cm},-0.95)}]A0);
	\coordinate (khTx) at ([shift={(0.18,-0.11)}]khT1);
	\coordinate (XM) at ([shift={({\lk/2},-2)}]A1);
	\coordinate (xzt) at ([shift={(0,-7.1)}]A1);
	\coordinate (fxzt) at ([shift={(0,-12.5)}]A1);
    \coordinate (txzt) at ([shift={(0,-15)}]A1);
	
	\draw[line width=1.2pt]
		(A2)rectangle(C2)
		(A3)rectangle(C3);
	\draw
		([shift={({\lk/2},0)}]A1)node[below]{\zihao{-4}%%PAPER_TITLE%%}
		([shift={({\lk/2},-0.55)}]A1)node[below]{\heiti\zihao{3}%%PAPER_SUBJECT%%\quad 答题卡};
	
	%%ANSWER_SHEET_CONTENT%%
\end{tikzpicture}

\newpage

\begin{tikzpicture}[remember picture,overlay]
	\coordinate (A1) at ([shift={(\zbj,-\sbj)}]current page.north west);
	\coordinate (B1) at ([shift={({\zbj+\lk},-\sbj)}]current page.north west);
	\coordinate (C1) at ([shift={({\zbj+\lk},\xbj)}]current page.south west);
	\coordinate (D1) at ([shift={(\zbj,\xbj)}]current page.south west);
	\coordinate (A2) at ([shift={(\ljjy,0)}]B1);
	\coordinate (B2) at ([shift={({\lk},0)}]A2);
	\coordinate (C2) at ([shift={(0,-\textheight)}]B2);
	\coordinate (D2) at ([shift={(\ljjy,0)}]C1);
	\coordinate (A3) at ([shift={(\ljje,0)}]B2);
	\path([shift={(-\ybj,-\sbj)}]current page.north east) coordinate (B3);
	\path([shift={(-\ybj,\xbj)}]current page.south east) coordinate (C3);
	\coordinate (D3) at ([shift={(\ljje,0)}]C2);
	
	\draw[line width=1.2pt]
		(A1)rectangle([shift={(-0.2,0)}]C1)
		(A2)rectangle(C2)
		(A3)rectangle(C3);
	\draw	[BrickRed,rounded corners=6pt]
		([shift={(-0.2,0.2)}]A1)rectangle([shift={(0,-0.5)}]C1)
		([shift={(-0.2,0.2)}]A2)rectangle([shift={(0.2,-0.5)}]C2)
		([shift={(-0.2,0.2)}]A3)rectangle([shift={(0.2,-0.5)}]C3);
	\draw([shift={({\lk/2},-0.5)}]D1)node[above=-1pt,BrickRed]{\heiti\zihao{6}请在各题对应答题区域内作答,超出矩形边框限定区域的答案无效};
	\draw([shift={({0.5*\lk},-0.5)}]D2)node[above=-1pt,BrickRed]{\heiti\zihao{6}请在各题对应答题区域内作答,超出矩形边框限定区域的答案无效};
	\draw([shift={({0.5*\lk},-0.5)}]D3)node[above=-1pt,BrickRed]{\heiti\zihao{6}请在各题对应答题区域内作答,超出矩形边框限定区域的答案无效};

%%SOLUTION_AREAS_PAGE2%%
\end{tikzpicture}
\end{document}
